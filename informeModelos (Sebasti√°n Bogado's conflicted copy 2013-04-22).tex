%%%%%%%%%%%%%%%%%%%%%%%%%%%%%%%%%%%%%%%%%%%%%%%%%%%%%%%%%%%%%%%%%%%%%%%%%%%%%%%
% Definici\'on del tipo de documento.                                           %
% Posibles tipos de papel: a4paper, letterpaper, legalpapper                  %
% Posibles tama�os de letra: 10pt, 11pt, 12pt                                 %
% Posibles clases de documentos: article, report, book, slides                %
%%%%%%%%%%%%%%%%%%%%%%%%%%%%%%%%%%%%%%%%%%%%%%%%%%%%%%%%%%%%%%%%%%%%%%%%%%%%%%%
\documentclass[a4paper,10pt]{article}


%%%%%%%%%%%%%%%%%%%%%%%%%%%%%%%%%%%%%%%%%%%%%%%%%%%%%%%%%%%%%%%%%%%%%%%%%%%%%%%
% Los paquetes permiten ampliar las capacidades de LaTeX.                     %
%%%%%%%%%%%%%%%%%%%%%%%%%%%%%%%%%%%%%%%%%%%%%%%%%%%%%%%%%%%%%%%%%%%%%%%%%%%%%%%

% Paquete para inclusi\'on de gr\'aficos.
\usepackage{graphicx}

\usepackage{enumerate}

% Paquete para definir la codificaci\'on del conjunto de caracteres usado
% (latin1 es ISO 8859-1).
\usepackage[latin1]{inputenc}

% Paquete para definir el idioma usado.
\usepackage[spanish]{babel}

\usepackage{multirow} 

% Paquete para f\'ormulas matem\'aticas
\usepackage{amsmath}

% Espacio luego de titulos
\usepackage{titlesec}

\newcommand{\BigO}[1]{\ensuremath{\operatorname{O}\bigl(#1\bigr)}}

%\titleformat{hcommandi}[hshapei]{hformati}{hlabeli}{hsepi}{hbefore-codei}[hafter-codei]
\titlespacing\subsection{5pt}{12pt plus 4pt minus 2pt}{12pt plus 2pt minus 2pt}
\titlespacing\subsubsection{10pt}{0pt plus 2pt minus 2pt}{0pt plus 2pt minus 2pt}



%\usepackage{multicolumn} 

% T\'itulo principal del documento.
\title{		\textbf{Trabajo pr\'actico}}

% Informaci\'on sobre los autores.
\author{	Alejandro Garc\'ia Marra, \textit{Padr\'on Nro. 91.516}                     \\
            \texttt{ alemarra@gmail.com }                                              \\
            Sebasti\'an Javier Bogado, \textit{Padr\'on Nro. 91.707}                     \\
            \texttt{ sebastian.j.bogado@gmail.com }                                              \\
            \normalsize{1er. Cuatrimestre de 2013}                       \\
            \normalsize{71.14 Modelos y Optimization I}                             \\
            \normalsize{Facultad de Ingenier\'ia, Universidad de Buenos Aires}            \\
       }
\date{}



\begin{document}

% Inserta el t\'itulo.
\maketitle

% Quita el n\'umero en la primer p\'agina.
\thispagestyle{empty}


\newpage
\section{Ejercicio Principal}

\subsection{An\'alisis del caso}

Una refiner\'ia tiene una producci\'on diaria de distintos tipos de combustibles y aceites a partir de petr\'oleo crudo de dos tipos distintos. A lo largo de los distintos procesos de destilaci\'on se consiguen los diversos productos. Cuanto m\'as refinado es el producto, mas valioso es para su venta.

\subsection{Objetivo}
Determinar la cantidad de barriles de los distintos tipos de combustible, fueloil y lubricante a producir, as\'i como la composici\'on de los combustibles para maximizar las utilidades de la refiner\'ia por d\'ia.
\vspace{10mm}

\subsection{Hip\'otesis y Aclaraciones}

\begin{itemize}
 \item {Precio constante en el d\'ia}
 \item {No tengo stock inicial}
 \item {Se vende todo lo producido y, por ende, se puede hablar de fracciones de barril}
 \item {Se dispone de dinero suficiente para comprar toda la materia prima necesaria}
 \item {Puedo comprar cantidades fraccionarias de la materia prima}
 \item {Las m\'aquinas no se rompen ni los empleados se rebelan}
 \item {Al hablar de ``barriles'' para refererirse a cantidad, un barril de un producto es igual al de otro}
 \item {No hay perdidas de producci\'on ni transporte, excepto las indicadas en la destilaci\'on}
 \item {En las mezclas no se agrega nada que no est\'e mencionado, entonces las proporciones deben sumar 1}
 \item {El centro de destilaci\'on puede alternar entre crudo de tipo 1 y tipo 2 sin p\'erdidas de tiempo o costos adicionales. Lo mismo aplica para el centro de reformado y craqueo respecto de sus distintas entradas}
 \item {A menos que se indique lo contrario, puede no producirse alguno de los productos finales}
 
 
 
\end{itemize}
	
\newpage

\subsection{Variables}


\subsubsection{Compra}
\vspace{5mm}

$C1 = $ Barriles de Crudo 1 por d\'ia \\

$C2 = $ Barriles de Crudo 2 por d\'ia \\
\vspace{2mm}

\subsubsection{Destilado}
\vspace{5mm}

$NL = $ Barriles de Nafta Liviana por d\'ia \\

$NM = $ Barriles de Nafta Mediana por d\'ia \\

$NP = $ Barriles de Nafta Pesada por d\'ia \\

$AL = $ Barriles de Aceite Liviano por d\'ia \\

$AP = $ Barriles de Aceite Pesado por d\'ia \\

$RDES = $ Barriles de Residuo Destilado por d\'ia \\
\vspace{2mm}

\subsubsection{Reformado}
\vspace{5mm}

$GREF = $ Barriles de Gasolina Reformada por d\'ia \\
\vspace{2mm}

\subsubsection{Craqueo}
\vspace{5mm}

$GCRA = $ Barriles de Gasolina Craqueada por d\'ia \\

$ACRA = $ Barriles de Aceite Craqueado por d\'ia \\
\vspace{2mm}

\subsubsection{Ventas}
\vspace{5mm}

$PR = $ Barriles de Combustible Premium por d\'ia \\

$SU = $ Barriles de Combustible Super por d\'ia \\

$AV = $ Barriles de Combustible para Aviones por d\'ia \\

$FO = $ Barriles de Fueloil por d\'ia \\

$LU = $ Barriles de Lubricante por d\'ia \\
\vspace{2mm}

\subsubsection{Otros Datos}
Aclaración: las siguientes son constantes
\begin{align*}
OCTNL &= 100 \ Octanos	& OCTREF &= 125 \ Octanos \\
OCTNM &= 90 \ \ Octanos	& OCTGCRA &= 115 \ Octanos \\
OCTNP &= 80 \ \ Octanos	& \\
\\
PRESAL &= 1 \ Kg/cm^{2} & PRESACRA &= 1.5 \ Kg/cm^{2} \\
PRESAP &= 0.6 \ Kg/cm^{2} & PRESRDES &= 0.05 \ Kg/cm^{2}
\end{align*}

\subsection{Ecuaciones}

\subsubsection{Destilado}

\begin{align*}
 NL &=   0,1 \ C1 \  + 0,15 \ C2   &   &= NL^{REF} + NL^{PR} + NL^{SU} \\
 NM &=   0,2 \ C1 \  + 0,25 \ C2   &   &= NM^{REF} + NM^{PR} + NM^{SU} \\
 NP &=   0,2 \ C1 \  + 0,18 \ C2   &   &= NP^{REF} + NP^{PR} + NP^{SU} \\
 AL &=   0,2 \ C1  		    &   &= AL^{AV} + AL^{CRA} + AL^{FO}\\
 AP &=   0,13 \ C1 \  + 0,12 \ C2  &   &= AP^{AV} + AP^{CRA} + AP^{FO} \\
 RDES &= 0,13 \ C1 \  + 0,12 \ C2  &  
\end{align*}

\subsubsection{Reformado}

\begin{align*}
 REF &=  0,6 \ NL^{REF} \  + 0,52 \ NM^{REF} + 0,45 \ NP^{REF} 
\end{align*}

\subsubsection{Craqueo}

\begin{align*}
 GCRA &=  0.28 \ AL^{CRA} \  + 0.2 \ AP^{CRA}  & &= GCRA^{PR} + GCRA^{SU}\\
 ACRA &=  0.68 \ AL^{CRA} \  + 0.75 \ AP^{CRA} & &= ACRA^{AV} + ACRA^{FO} \\
 \end{align*}

\subsubsection{Ventas}
 
\begin{align*}
 PR &=  NL^{PR} + NM^{PR} + NP^{PR}+ GRCA^{PR} + REF^{PR} \\
 SU &=  NL^{SU} + NM^{SU} + NP^{SU}+ GRCA^{SU} + REF^{SU} \\
 18 \ FO &=  10 \ AL^{FO} + 3 \ AP^{FO} + 4\ ACRA^{FO}+ 1\ RDES^{FO} \\
 AV &=  AL^{AV} + AP^{AV} + ACRA^{AV} + RDES^{AV} \\
 LU &=  RDES^{LU} 
 \end{align*}

\subsubsection{Funcional}

\begin{align*}
 Z_{max} &= 700 \ \text{\$/B} \cdot \ PR + 600\ \text{\$/B} \cdot \ SU + 400\ \text{\$/B} \cdot \ AV + 350\ \text{\$/B} \cdot \ FO + 150\ \text{\$/B} \cdot \ LU \\
\end{align*}

\subsection{Restricciones}

\begin{align*}
 C1 &\leq 20000 \ B/d \\
 C2 &\leq 30000 \ B/d \\
 C1 + C2 &\leq 4500 \ \ B/d \\
 \\
 NL^{REF} + NM^{REF} + NP^{REF} &\leq 1000 \ B/d \\
 AL^{CRA} + AP^{CRA} &\leq 8000 \ B/d \\
 \\
  500 B/d \ \leq\ &LU \leq \ 1000 \ B/d \\
 \\
PR &\geq 0.4 \ SU 
 \end{align*}
 \begin{align*}
PR \cdot 98 \ Oct \ &\geq OCTNL \cdot NL^{PR} + OCTNM \cdot NM^{PR} + OCTNP \cdot NP^{PR} +  \\
			& \ \ \ OCTCRA \cdot GCRA^{PR} + OCTREF \cdot REF^{PR}\\
\\
SU \cdot 95 \ Oct \ &\leq OCTNL \cdot NL^{SU} + OCTNM \cdot NM^{SU} + \\
& \ \ \ OCTNP \cdot NP^{SU} + OCTCRA \cdot GCRA^{SU} + OCTREF \cdot REF^{SU} \ \leq SU \cdot 97,99 \ Oct  \\
\\
AV \cdot 1 \ Kg/cm^{2} &\geq PRESAL \cdot AL^{AV} + PRESAP \cdot AP^{AV} + \\
				& \ \ \ PRESACRA \cdot ACRA^{AV} + PRESRDES \cdot RDES^{AV}\\
\end{align*}

\newpage

\section{Ejercicio Complementario}

\subsection{An\'alisis del caso}

Pepe desea fabricar sidra de manera artesanal para luego venderla a una cadena re locales gourmet. Para esto, cuenta con tres tipos distintos de manzanas como materia prima, a partir de las cuales se pueden obtener tanto sidra natural como sidra dulce. La diferencia entre una y otra surge de cambios en la proporci\'on de las diferentes manzanas utilizadas. 

\subsection{Objetivo}

Determinar la cantidad de sidra de uno y otro tipo a producir de forma tal de maximizar las ganancias que obtiene Pepe en un per\'iodo. (En este caso utilizaremos un per\'iodo de una semana, ya que lo consideramos razonable.)

\subsection{Hip\'otesis}
\begin{itemize}
  \item{Dispone del dinero para comprar toda la materia prima necesaria}
 \item{Dispone de tiempo suficiente para realizar toda la producci\'on indicada}
 \item{No tiene stock previo}
 \item{La cadena de locales comprar\'a toda la producc\'on obtenida}
 \item{No hay p\'erdidas de materia prima en el proceso ni en el transporte}
 \item{No hay p\'erdidas de producci\'on}
 \item{Los precios son constantes dentro del per\'iodo.}
\end{itemize}
 
\subsection{Variables}

 \subsubsection{Compra}
\vspace{2mm}
 $ R = $ Kg/Sem de Manzanas Rojas
 
 $ V = $ Kg/Sem de Manzanas Verdes
 
 $ A = $ Kg/Sem de Manzanas Amarillas
\vspace{2mm}

 \subsubsection{Venta}
\vspace{2mm}
 $ SN = $ Litros/Sem de Sidra Natural
 
 $ SD = $ Litros/Sem de Sidra Dulce
 \vspace{2mm}
 
 \subsubsection{Funcional}
 \vspace{5mm}
 $ING =$ Ingresos por ventas semanales
 
 $Z \xrightarrow{}$ MAX $ING$
 \vspace{2mm}
 
\subsection{Ecuaciones}
\begin{align*}
 SN &= 2\ L/Kg \cdot \ V + 2 \ L/Kg \cdot \ A + 1 \ L/Kg \cdot \ R \\
 SD &=  1 \ L/Kg \cdot \ V + 1 \ L/Kg \cdot \ R \\
 \\
 Z &= 20 \ \$/L \cdot \ SN + 15 \ \$/L \cdot \ SD 
\end{align*}

\subsection{Restricciones}
\begin{align*}
 V &\leq 15 Kg/Sem\\
 R &\leq 60 Kg/Sem\\
 A &\leq 15 Kg/Sem\\
 \\
 SD &\geq 10 L/Sem
\end{align*}


 
 




\end{document}
